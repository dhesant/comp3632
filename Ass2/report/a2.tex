\documentclass[12pt,a4paper]{article}

\usepackage{float}
\usepackage[margin=1in]{geometry}
\usepackage {mathtools}
\usepackage {amssymb}

% Fix URLs
\usepackage{url}
\usepackage[pdfborder={0 0 0},breaklinks=true]{hyperref}
\usepackage{breakurl}
\urlstyle{same}  % don't use monospace font for urls

\title{COMP3632 - Assignment 1}
\author{Dhesant Nakka\\20146587}


\begin{document}
\maketitle

\section{Question 1}
\subsection{Part a}
\subsection{Part b}

\section{Question 2}
IP spoofing \(\rightarrow\) Ingress/egress filtering: Ingress/egress filtering is a defense against IP spoofing because it filters out packets who's IP address are outside a specific range, for example, an allowed range defined by a sysadmin, or the range for which the upstream source has been assigned IP addresses.\\
Eavesdropping \(\rightarrow\) Proxies: Proxies are a defense against eavesdropping by creating a secure tunnel where all the requests are forwarded out of the vunerable network towards the proxy. Since the proxy will be used by multiple people, it is almost impossible to deteremine which user made which request when viewed downstream of the proxy.\\
Teardrop attack \(\rightarrow\) Deep Packet Inspection: Deep Packet Inspection (DPI) is a defense against Teardrop attacks. DPI scans each packet, and ensures that it has the correct format and is not mangled in any way, preventing any malformed packet from being sent to a server where it can cause problems.

\section{Question 3}
\subsection{Part a}
\(K_1\) is a private key.\\
\(K_2\) is a public key.\\
\(K_3\) is a secret key.

\subsection{Part b}
\(K_1\) \& \(K_2\) are long-lasting (compared to \(K_3\), however, \(K_1\) \& \(K_2\) should also be changed periodically to prevent future attacks.)\\
\(K_3\) is ephemeral.

\subsection{Part c}

\section{Question 4}
\subsection{Part a}
Let \(K\) be the one-time pad encryption key. Therefore, \(C_1 = P_1 \oplus K\), and \(C_2 = P_2 \oplus K\).
Therefore, to obtain \(P_1 \oplus P_2\);
\begin{align*}
  C_1 \oplus C_2 &= (P_1 \oplus K) \oplus (P_2 \oplus K)\\
  &= P_1 \oplus K \oplus P_2 \oplus K\\
  &= (P_1 \oplus P_2) \oplus (K \oplus K)\\
  &= P_1 \oplus P_2
\end{align*}

\subsection{Part b}
To obtain \(P_1\) and \(P_2\), we can use a technique called crib-dragging. This is where some guess for \(P_1\) is XOR'ed with the result from part a (\(P_1 \oplus P_2\)). This results in;
\[ P_1 \oplus (P_1 \oplus P_2) = P_2 \]
Based on that equation, we can keep trying various words for \(P_1\), and XOR'ing them with the result from part a, until the output in \(P_2\) resembles what we expect, i.e. part of an English word/sentence. We can continue extending the guesses for both \(P_1\) and \(P_2\) until the full plain-texts are found.

\section{Question 5}
\subsection{Part a}
Nodes do not want to be an exit node because all the traffic generated by any user that uses the exit node is viewed as coming from the exit node to the rest of the internet. This causes liability issues for the exit node provider, since any illegal activity will be recorded as coming from them, and they are then faced with the burden of offloading the liability to the responsible party.

\subsection{Part b}


\subsection{Part c}
An advantage of using three nodes instead of five is latency, as the packet does not have to make as many stops in a three node setup as it does in a five node setup, and with fewer stops, the packet takes less time to travel to its destination.

\subsection{Part d}
Using 3 nodes increases the level of privacy for the user, since in a one node setup, the node provider is able to see who is connecting to the node, and who the node is connecting to, and can therefore determine what each user is connecting to.

\subsection{Part e}


\section{Padding Oracle Attack}
\subsection{Part a}
\[
  x' = (\mathtt{ABABABABABOAOAOAOAOAOAOAOAOAOAOA})_{16}
\]

\subsection{Part b}
\begin{align*}
  x_1 &= D(y_1) \oplus IV \\
  x_2 &= D(y_2) \oplus y_1
\end{align*}

\end{document}
