\documentclass[12pt,a4paper]{article}

\usepackage{float}
\usepackage[margin=1in]{geometry}
\usepackage {mathtools}
\usepackage {amssymb}

% Fix URLs
\usepackage{url}
\usepackage[pdfborder={0 0 0},breaklinks=true]{hyperref}
\usepackage{breakurl}
\urlstyle{same}  % don't use monospace font for urls

\title{COMP3632 - Assignment 1}
\author{Dhesant Nakka\\20146587}


\begin{document}
\maketitle

\section{Question 1}
\subsection{Part a}
The problem with this situation is the use of ECB mode in AES, which is highly insecure, as identical plaintexts blocks result in identical ciphertext blocks, and is therefore susceptible to various attacks, such as a frequency attack. Instead, a different AES mode should be used, such as CBC or CTR mode, as each ciphertext block is unique, even if the plaintext blocks are identical.

\subsection{Part b}
The problem with this situation is that the checksum performed on the ciphertext, and is sent as plaintext. This allows an attack who is performing a man-in-the-middle attack to replace the message with their own malicious message, and recalculate the checksum based on the new ciphertext, before sending it to Bob, who cannot tell the difference between Alice's message and the attacker's message. Instead, the checksum should be performed on the plaintext, and encrypted along with the plaintext to form the ciphertext that is sent, so that any modifications to the ciphertext result in the checksum being invalid for the decoded plaintext.


\section{Question 2}
IP spoofing \(\rightarrow\) Ingress/egress filtering: Ingress/egress filtering is a defense against IP spoofing because it filters out packets who's IP address are outside a specific range, for example, an allowed range defined by a sysadmin, or the range for which the upstream source has been assigned IP addresses, preventing the spoofed packets from being received by the target.\\

Eavesdropping \(\rightarrow\) Proxies: Proxies are a defense against eavesdropping because it conceals either the source or the destination for any packet, depending on whether the attacker is downstream or upstream of the proxy respectively. By using a proxy, an attacker is unable to determine which user sends which packet to which destination, especially if encryption is added so the contents of packet, and as such, prevents eavesdropping.\\

Teardrop attack \(\rightarrow\) Deep Packet Inspection: Deep Packet Inspection (DPI) is a defense against Teardrop attacks. DPI scans each packet, and ensures that it has the correct format and is not mangled in any way, preventing any malformed packet from being sent to a server where it can cause problems.

\section{Question 3}
\subsection{Part a}
\(K_1\) is a public verification key.\\
\(K_2\) is a public encryption key.\\
\(K_3\) is a secret key.

\subsection{Part b}
\(K_1\) is long-lasting.
\(K_2\) \& \(K_3\) are ephemeral.

\subsection{Part c}
To ensure that \(K_1\) is correct, the client uses a certificate authority which it trusts. The trusted certificate authority signs the public key \(K_1\), creating a certificate. This certificate is then sent along with the public key during the first step of TLS authentication. When the client receives the public key and the certificate, it can use it's built-in collection of trusted certificate authority keys to verify the public key against the certificate, ensuring it is correct.

To ensure that \(K_2\) is correct, it is sent from the server to the client along with a signature that is generated by signing \(K_2\) with the private key for \(K_1\). Once the client has verified that \(K_1\) is accurate (see above), it can use \(K_1\) to verify \(K_2\) by using the \(K_1\) to decode the signature, and ensure that this signature matches the key \(K_2\), thus ensuring \(K_2\) is accurate.

\section{Question 4}
\subsection{Part a}
Let \(K\) be the one-time pad encryption key. Therefore, \(C_1 = P_1 \oplus K\), and \(C_2 = P_2 \oplus K\).
Therefore, to obtain \(P_1 \oplus P_2\);
\begin{align*}
  C_1 \oplus C_2 &= (P_1 \oplus K) \oplus (P_2 \oplus K)\\
  &= P_1 \oplus K \oplus P_2 \oplus K\\
  &= (P_1 \oplus P_2) \oplus (K \oplus K)\\
  &= P_1 \oplus P_2
\end{align*}

\subsection{Part b}
To obtain \(P_1\) and \(P_2\), we can use a technique called crib-dragging. This is where some guess for \(P_1\) is XOR'ed with the result from part a (\(P_1 \oplus P_2\)). This results in;
\[ P_1 \oplus (P_1 \oplus P_2) = P_2 \]
Based on that equation, we can keep trying various words for \(P_1\), and XOR'ing them with the result from part a, until the output in \(P_2\) resembles what we expect, i.e. part of a word/sentence. We can continue extending the guesses for both \(P_1\) and \(P_2\) until the full plaintexts are found.

\section{Question 5}
\subsection{Part a}
Nodes do not want to be an exit node because all the traffic generated by any user that uses the exit node is viewed as coming from the exit node to the rest of the internet. This causes liability issues for the exit node provider, since any illegal activity will be recorded as coming from them, for which they are responsible for, unless they can obtain some agreement with the corresponding law enforcement agency which absolves them from the liability.

\subsection{Part b}
The final node in the Tor network can read $C$'s private information. This is because for each node of the Tor network, a layer of encryption is added, so none of the preceding nodes are able to read the information, but the final node is responsible for transmitting the unencrypted information to the destination, and thus has access to it.

\subsection{Part c}
An advantage of using three nodes instead of five is latency, as the packet does not have to make as many stops in a three node setup as it does in a five node setup, and with fewer stops, the packet takes less time to travel to its destination.

\subsection{Part d}
Using 3 nodes increases the level of privacy for the user. Typically, the first node in the Tor circuit knows who you are, and the final node in the Tor circuit knows who you are connecting to. In a three node circuit, the first and final nodes are different nodes, and therefore do not have access to both items of data.. In a one node circuit, the node knows both who you are, and who you are connecting to, which can cause privacy issues for the user.

\subsection{Part e}
An advantage of using three nodes instead of two in a Tor network is privacy. With only two nodes in a circuit, the first node knows the source of the packet, and the fact that the packet will be routed to the destination from the second node, and the second node knows the destination of the packet, and that the packet's source is connected to the first node. Using this information, it is possible to correlate the traffic to figure out the source and destination of individual packets. In a three node setup, there is a buffer between the first and last node, removing the knowledge of which node the source/destination is connected to, as the intermediate node can both receive and transmit data from various other nodes at the same time.

\section{Padding Oracle Attack}
\subsection{Part a}
\[
  x' = (\mathtt{ABABABABAB0B0B0B0B0B0B0B0B0B0B0B})_{16}
\]

\subsection{Part b}
\begin{align*}
  x_1 &= D(y_1) \oplus IV \\
  x_2 &= D(y_2) \oplus y_1
\end{align*}

\end{document}
