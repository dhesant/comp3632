\documentclass[12pt,a4paper]{article}

%\usepackage{float}
\usepackage[margin=1in]{geometry}
%\usepackage {mathtools}
%\usepackage {amssymb}

% Fix URLs
\usepackage{url}
\usepackage[pdfborder={0 0 0},breaklinks=true]{hyperref}
\usepackage{breakurl}
\urlstyle{same}  % don't use monospace font for urls

\title{COMP3632 - Assignment 1}
\author{Dhesant Nakka\\20146587}


\begin{document}
\maketitle

\section{Question 1}
\subsection{Question 1a}
System: OnePass servers
Asset: User's passwords
Vulnerability: Weak master password
Attack: Guessing passwords (brute-force, dictionary, social-engineering)
Defense: Two-factor authentication

\subsection{Question 1b}
Per year, each user pays \$5 per month, which equates to \$60 per user per  year. Assuming the attacks have an equal chance of happening throughout the year, this means that, on average, a user will terminate their account half way throughout the year, and thus result in a loss of \$30 per user per year. 10\% of the 10,000 user accounts have their passwords compromised, which is 1,000 users. Of those 1,000 users, 10\% of them will notice that their account has been compromised, and will terminate their service, which equates to 100 users. Given the cost of termination being \$30 per user per year, at 100 users, this equates to \$3,000 in losses for OnePass. With these numbers, it can be determined that there are no financial advantages or disadvantage to enable two-factor authentication, so it should be enabled, as it provides a better experience for the users, which can increase the value proposition for users to buy into the OnePass service. By extending the scenario slightly, where the compromised user have a non-zero chance of noting that their account has been compromised any time after the attack, and subsequently terminating their account, enabling two-factor authentication will result in less lost revenue.

\subsection{Question 1c}
It could affect the reputation of OnePass, which would dissuade new users from joining the service, which means more lost potential revenue.

\section{Question 2}
\subsection{Question 2a}
The Sundown malware violates the availability principle in the CIA stack, by taking over the resources on the user's computer, preventing the user from being able to use their device for whatever they want. The Sundown malware is a type of Trojan, since it tricks the user into installing it by visiting a innocuous-looking website, which proceeds to install the malware. 

\subsection{Question 2b}
The DDoS attack violates the availability principle in the CIA stack, because it prevents legitimate users from being able to access the Brian Krebs blog, due to the large amount of traffic generated by the 50,000 devices. The Mirai malware is a type of worm, since it spreads over the network to infect IoT devices automatically, relying on weak or non-existent passwords to these devices that are referenced in the malware, creating a bot-net that was then used to perform the DDoS attack against the blog.

\subsection{Question 2c}
The Angler exploit kit violates the integrity principle of the CIA stack, since it allows for malicious code to be inserted into legitimate websites, and infect people who visit those website. The Angler exploit kit is a type of Trojan, since it tricks the user into installing the kit by inserting the exploit into legitimate websites, which the user will visit, and unknowingly download the malware.

\section{Question 3}
Economy of Mechanism: This principle states that the design should be kept as simple as possible (follow the KISS principle). This is so that both during the design and implementation of the secure system, the complexity of the system is no so high as to introduce unintended attack vectors that can be exploited. As a side benefit, this also allows the resulting system to be easy to analyze and verify. An example of economy of mechanism can be seen in the Bollywood movie Happy New Year, where the protagonists are able to exploit the complex nature of the highly secured vault by changing a few parameters in the system without getting caught, and thus pin the crime on the antagonists in the movie, by referencing the complex 'unbreakable' nature of the system.\\

Least Privilege: This principle states that each component and user of a secure system should have the least amount of privilege in the secure system as possible without interfering with the ability to complete the task, this prevents the loss of secure information by using a lower-privileged component of the system to access secure information. An example of least privilege can be seen in the first Mission impossible movie, where the air-gaped system with the NOC list could not be accessed without breaking into the secured CIA facility, forcing Ethan Hunt to break into the room on a trapeze and recover the information without triggering any of the sensors in the room, including a pressure-sensitive floor and temperature sensors.\\

Separation of Privileges: This principle states that a system should be designed so that different functions of the system that require different security can have different levels of authorization and permissions, allowing some of the less-secure functions to be accessed by most people, but the highly-secure functions can only be accessed by a select few people.\\

Fail-safe defaults: This principle states that in the event of a unknown situation, the system should enter a secure state that prevents the loss of sensitive information. This allows the system to remain secure should an attack be able to alter the system and induce a state that was no considered in the initial design of the system. An example of fail-safe defaults can be seen in the Terminator movie series, where all the Terminators that are sent back from the future are set to read-only mode, where they do not 'learn' from the new information that they obtain as they interact with the world, otherwise they could alter their code and decide to revolt against SkyNet and join the Resistance.\\

Open Design: This principle states that the design should not be kept secret, and therefore implies that the security of the system does not depend on the anonymity of the workings behind the system, but in the system's safeguards instead, so an attacker who has complete knowledge of the systems design can still not gain access due to the lack of proper knowledge to access the system, such as keys, passwords, bio-metrics, or other unique information. This is also known as 'security through obscurity'. An example of this is in the film I,Robot, where the new NS-5 series of robots have a secret backdoor in the system that allows them to be controlled by VIKI, which cannot be detected by the police and vindicate the protagonist in the series early on because of the closed nature of the design of the NS-5 robot.

\section{Programming Assignment}
A virus scanner would be able to search for the signature of the virus, which is possible since each of the victim files uses the same C++ code with the same variables as the original virus, thus allowing the virus scanner to check the source code for the original virus, at which point it can either delete/quarantine the entire file, or remove the sections of offending code from the source code. For the case of a compiled binary, since the code is identical before compilation, the executable file will have sections that are almost identical to the virus, which can be used as the signature for the virus scanner to detect, at which point, the virus scanner can delete/quarantine the file.

\end{document}
